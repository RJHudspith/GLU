\documentclass[12pt]{article} 

\usepackage[english]{babel}
\usepackage{times}
\usepackage[T1]{fontenc}
\usepackage{blindtext}
\usepackage{mathptmx}

% the following commands control the margins:
\topmargin=-1in    % Make letterhead start about 1 inch from top of page 
\textheight=20cm  % text height can be bigger for a longer letter
\oddsidemargin=0pt % leftmargin is 1 inch
\textwidth=6.5in   % textwidth of 6.5in leaves 1 inch for right margin
\usepackage{hyperref}

\begin{document}

\section{Introduction}

Hi, I am producing this document to ease you in to using my code/library, which I have called ``GLU'' (Gauge Link Utility). The code is intended for very fast Landau and Coulomb gauge fixing using Fourier
Acceleration \cite{PhysRevD.37.1581}. It also supports a bunch of other operations which I will not describe here.

My code is written for \href{http://en.wikipedia.org/wiki/Openmp}{openmp shared-memory parallelism}. And can read
\href{http://lattices.qcdoc.bnl.gov/formatDescription/index.html}{NERSC}, HiRep, MILC, SCIDAC, \verb|ILDG_BQCD|, \verb|ILDG_SCIDAC| or general LIME configuration files.

I have included the most recent \href{http://www.fftw.org/}{fftw} library, this is the only external library needed. I
will now describe the fairly painless installation procedure, which should take five minutes or so.

\section{Installation}

Ok, so I will discuss the two choices in the utilisation of FFTW, they are the ``Small scale'' and ``Large scale''
usage. ``Small scale'' is for targets of less than 6 processors and ``Large scale'' is for more. My code can be set to
generic NC (number of color charges) and generic ND (dimensions of lattice), but defaults to NC=3, ND=4 which is what I
imagine you require.

\subsection{FFTW and flags}

In fftw-3.3.3 the general flags I used are (assuming gcc and \verb| $FFTW_INSTALL | is some meaningful directory),
\small{
\begin{verbatim}
./configure CC=gcc CFLAGS="-O3 -ffast-math" --prefix=$FFTW_INSTALL 
\end{verbatim}
}

And if available, I would recommend using the \verb| -mfpmath=sse | and the \verb| --enable-sse2 | flags as well (if
available). And then it is just the usual \verb| make && make all install | 

\subsubsection{Large scale}

For large scale builds I utilise the OpenMP'd FFTW routines, called using

\begin{verbatim}
 --enable-openmp
\end{verbatim}

and \verb| -fopenmp | in the CFLAGS. 

\subsection{GLU and flags}

Awesome, you have compiled FFTW! Don't worry, that was the hard part. Now in ./GLU/, type something like

\small{
\begin{verbatim}
./configure CC=gcc CFLAGS="-O3 -ffast-math -fopenmp" --prefix=$GLU_INSTALL \\
--enable-timegf --with-fftw=$FFTW_INSTALL
\end{verbatim}
}

Where the \verb|\\| implies I have not gone to a newline, this was just for formatting in Latex.

And then just \verb| make && make install |.

A note of caution, please make sure \verb| $GLU_INSTALL | is the full path, because we have the option to save some
time and this option will be discussed in ``Erroneous''.

\subsubsection{Large scale}
 
As with the fftw, we need to state we are using the openMP'd fftw routines to my code. The configure flag
\begin{verbatim}
 --enable-OMP_FFTW
\end{verbatim}
does the trick.

\section{Running tests}

Fantastic. So you should notice there is now something in the \verb| $GLU_INSTALL  | directory. Should be ``bin, lib,
share and Local''. In the bin directory you will find the binary
GLU. A flat input file and a config (NERSC QCD
configuration ``lat..400''). You run my code by typing ( the terms in {} are optional as to whether you want an output
file).

\begin{verbatim}
 ./GLU -i input_file -c lat..400 {-o $outfile}
\end{verbatim}

The input file given is set-up for Coulomb gauge fixing, the accuracy is some integer in the file which is converted to
$10^{-\textrm{ACCURACY}}$ in the code, due to historical reasons. The rest of the stuff in there is for the
most part unimportant. If you are not converging set ``\verb|GF_TUNE|'' to be lower, say ``0.09''.

The accuracy is defined by the measure,
\begin{equation}
\theta=\frac{1}{V N_C}\sum_{\mu,x} Tr\left(\left[ A_\mu(x+a\hat{\mu}/2) -
A_\mu(x-a\hat{\mu}/2)
\right]\right)^2 < 10^{-ACCURACY}. \nonumber
\end{equation}

\textbf{I strongly recommend an ACCURACY greater than or equal to 14!}

GLU can also be used to automatically generate example input files for Landau and Coulomb gauge fixing by typing the command,
\begin{verbatim}
./GLU --autoin={LANDAU,COULOMB}
\end{verbatim}
This will print out an input file to stdout. If you wish to query the various options of the input file, the command,
\begin{verbatim}
./GLU --help={options}
\end{verbatim}
Can be used.

\section{Erroneous}

If we are not submitting to a cluster and working locally, we can save the FFTW plan to the ./Local/Moments folder in
the installation directory. For large transforms this is a good initial time saver. It is enabled with the configure
flag,
\begin{verbatim}
--enable-notcondor
\end{verbatim}

For the moment that is about it.

\begin{thebibliography}{9}
  
\bibitem{PhysRevD.37.1581}
  \emph{Fourier acceleration in lattice gauge theories. I. Landau gauge fixing},
  Davies, C. T. H. and Batrouni, G. G. and Katz, G. R. and Kronfeld, A. S. and Lepage, G. P. and Wilson, K. G. and Rossi, P.  and Svetitsky, B.,
  Phys. Rev. D,
  37,
  6,
  1581--1588,
  7,
  1988,
  Mar,
  10.1103/PhysRevD.37.1581,
  American Physical Society.

\end{thebibliography}

\end{document}
