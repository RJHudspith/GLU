\documentclass[12pt]{article} 

\usepackage[english]{babel}
\usepackage{times}
\usepackage[T1]{fontenc}
\usepackage{blindtext}
\usepackage{mathptmx}
\usepackage{amsmath,mathtools,amsfonts}

% the following commands control the margins:
\topmargin=-1in    % Make letterhead start about 1 inch from top of page 
\textheight=20cm  % text height can be bigger for a longer letter
\oddsidemargin=0pt % leftmargin is 1 inch
\textwidth=6.5in   % textwidth of 6.5in leaves 1 inch for right margin
\usepackage{hyperref}

\begin{document}

Hi, this short document discusses the computation of $SU(NC)\times U(1)$ configurations, where the U(1) is quenched, i.e. No dynamical fermion loops within travelling photons.

This option is turned on by the option in the input file,
\begin{verbatim}
 MODE = SUNxU1
\end{verbatim}

\section{What the code does}

The code computes a non-compact Gaussian-distributed Feynmann-gauge U(1) configuration by Fourier transform. It does so by having the complex-valued momentum space configurations gaussian distributed with $P,-P$ symmetry. It divides the fields by $\frac{1}{\sqrt{\beta}}\sqrt{p_\mu p_\mu}$ and Fourier transforms them to configuration space.

The coupling $\beta$ is related to the coupling value $\alpha$ by,
\begin{equation}
 \beta = \frac{1}{(ND)\pi \alpha} 
\end{equation}
The value $\alpha$ is set in the input file with the flag,
\begin{verbatim}
 U1_ALPHA = {}
\end{verbatim}

You can also change the charge of the U(1) field ``e''. Set in the input file with,
\begin{verbatim}
 U1_CHARGE = {}
\end{verbatim}

What we end up with is,
\begin{equation}
 U_\mu(x) = e^{ie\frac{1}{\sqrt{\beta}}A_\mu(x)}
\end{equation}
Where $A_\mu(x)$ is the non compact field.

We must then compactify the (Real) non-compact gauge-fixed fields to U(1) by exponentiation. We then multiply each polarisation of SU(NC) gauge field with the compact U(1) field. This overwrites the SU(NC) gauge field, and it is no-longer SU(NC) but rather U(NC).

\subsection{Measurements}

The U(1) field measurements are set in the input file by the tag,
\begin{verbatim}
 U1_MEAS = {U1_PLAQUETTE,U1_RECTANGLE,U1_TOPOLOGICAL}
\end{verbatim}
The measurement \verb\U1_PLAQUETTE\ is the default behaviour and measures the non compact plaquette and the compact plaquette. The rectangle measurement computes the 2x1 compact rectangle coefficient. And \verb\U1_TOPOLOGICAL\ computes some weird stuff like the number of windings of the noncompact field.

\textbf{Warning}

I imagine you will want to write the fields to some output file. That's cool. The SU(NC)xU(1) fields are Unitary, not special unitary. The only available NERSC output type is the \verb\NCxNC\ type (SCIDAC, ILDG_SCIDAC, ILDG_BQCD, HIREP and MILC save the whole matrix). The code uses will change to the applicable NERSC variant automatically.

\end{document}
